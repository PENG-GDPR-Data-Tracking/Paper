\subsection{Analysis and viability of technologies}
%  Which tech do we use? What is it? 
% Identify and technically describe available technologies that might be used (if any)



OpenApi is the most common standard for \ac{api} descriptions defining the interface between services. Most of the other standards are providing migrations or compatibility to the OpenAPI standard \cite{Scherer2016DescriptionTransformation} 
% https://github.com/oasis-tcs/odata-openapi


It is supported by the OpenApi initative, which is joined by major companies like Microsoft, Google, \cite{TheLinuxFoundation2020CurrentInitiative}
The standard developed from the Swagger Specification and is now almost 9 years old. 

OpenAPI is partly based on JSON SChema language which describes JSON Payload 



For our prototype we are using NodeJS Services

based on Express 

Javascript/Typescript Language for object oriented programmming which provides both typed experience and untyped functions. This will speed up and make. 

As the prototype will not reach production status we do no not need to factor in runtime performances. 

\subsection{Short Introduction to Technologies}




% explain OpenAPI (maybe some history of swagger as well, but not too lengthy) 
% maybe the differences between Swagger(v2) and OpenAPI(v3)? JSONSchema 
% Computer-readable format is needed to be make automated (or semi-automated) processing possible -> OpenAPI is a standard allowing us to describe APIs
% explain tracing frameworks like Zipkin and Jaeger
% explain static analysis
% runtime variables in cloud, eg. location (AWS has variables) 

\subsection{Short Introduction to Technologies}
% Assess practical viability of said technologies



% formulate why we used static analysis or tracing frameworks and which tradeoffs these have
% maybe some usage statistics about OpenAPI / tracking / other tools? How many developers are using it? Is it easy to implement? 
% Why these solutions? What are alternatives and why did we not choose them? 
% Ranking of Tools? 
% How much effort is it? 

% OpenAPI is very versatile: Code can be annotated to generate the specification, but there are also Code generation solutions that generate code from the spec itself.


\subsection{Prototype}
% Provide a (sketchy) outlook about what you are going to implement (esp. your re-usable component)

As described in section \ref{} (Problem) we want to make it easier to track the processing of data throughout applications. We based this on defining the type of data in a central place, which we identiefied as OpenAPI. In order to track data inside of the application we also need a service middleware, which enriches a request with the legal context and logs the type of data transferred. 
Collecting the 
Our Prototype will be set up in a business context. It will simulate a sleep tracking app. Which collects different categories of personal data for differents purposes, while beeing based on a microservice architecture. 

It will roughly look like the system shown in Figure \ref{fig}

\begin{figure}[htbp]
\centerline{\includegraphics{fig1.png}}
\caption{Sketch of our prototype system}
\label{fig}
\end{figure}

Our reusable software components can be split into three main areas: 

\begin{itemize}
    \item Schema Specifications
    \item Service Middleware 
    \item data collector 
\end{itemize}



This could be extended by providing decorators for easier schema specifactions or visualizations for the output data. 
% - Diagramm sketch 
% - flow diagram 

% - Reusable Component: 
%    - OpenAPI Specification based on "Transparency Tracing v2"
%    - Middleware for adding purpose to requests 
%       https://rhonabwy.com/2019/01/06/adding-tracing-with-jaeger-to-an-express-application/
%    - Linter for checking those: https://palantir.github.io/tslint/develop/custom-rules/
%     % https://medium.com/@andrey.igorevich.borisov/writing-custom-tslint-rules-from-scratch-62e7f0237124
%    - Zero Config - Middleware -> % should log to localhost by default? 
    % https://expressjs.com/en/guide/writing-middleware.html
    
    
%    - Visualization of said trace 
%    - Decorator pattern for adding specification easily to OpenAPI % and enable logging itself?
%    - static analysis tool for tracing the data flow

Goals: 
 - address that services are mostly not microservices in the sense of they fulfill only one purpose (in a GDPR sense) but are fulfilling multiple purposes. e.g. a (example from our system) This has to be taken care of at the application level, for each data request) 
 
 Because there are a multitude of different languages and each langauge has multiple frameworks and tools for making web requests or fetching data we propose a set of easy linting rules which can be applied to each request made and rewritten. 
 
 
 Another pain point is 